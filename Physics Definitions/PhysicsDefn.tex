\documentclass[a4paper]{article}

\usepackage[a4paper]{geometry}
\usepackage{amsmath}
\usepackage{amsbsy}
\usepackage{libertine}
\usepackage{parskip}
\usepackage{booktabs}
\usepackage[usenames,dvipsnames,svgnames,table]{xcolor}
\usepackage{makecell}
\usepackage{hyperref}
\hypersetup{pdftitle={The Definitive Physics Definition List},pdfauthor={Engineers of Dubious Quality},bookmarksnumbered=true,bookmarksopen=true,bookmarksopenlevel=1,colorlinks=true,allcolors=black,pdfstartview=Fit,pdfpagemode=UseOutlines,pdfpagelayout=TwoPageRight}

\newlength{\oldparskip}
\setlength{\parskip}{2ex plus 0.5ex minus 0.2ex}

\newcolumntype{L}[1]{>{\raggedright\let\newline\\\arraybackslash\hspace{0pt}}p{#1}}
\newcolumntype{C}[1]{>{\centering\let\newline\\\arraybackslash\hspace{0pt}}p{#1}}
\newcolumntype{R}[1]{>{\raggedleft\let\newline\\\arraybackslash\hspace{0pt}}p{#1}}

\title{The Definitive Physics Definition List}
\author{Engineers of Dubious Quality}

\begin{document}
	
	\maketitle
	
	\section{Measurements}
	Express errors/uncertainties to 1 s.f. and write the measured value to the same decimal place as its error/uncertainty
	
	\begin{center}
		\renewcommand{\arraystretch}{1.2}
		\begin{tabular}{@{} l p{10.5cm} @{}}
			\toprule
			Systematic Error & An error that occurs consistently more or consistently less than the actual reading.\\
			Random Error & An error that occurs as a scattering (or spreading) of readings about the average or mean value of the measurements. \\
			\midrule
			Precision & The \textit{\textbf{reproducibility}} of a measurement. Repeated measurements which are very close to one another are precise measurements. Thus an experiment which has \textit{\textbf{small random errors}} (i.e. small spread of readings) is said to have \textit{\textbf{high precision}}. \\
			Accuracy & The \textbf{\textit{agreement}} between the measured value and the true or accepted value of a quantity. An experiment which has \textbf{\textit{small systematic errors}} is said to have \textbf{\textit{high accuracy}}. The \textbf{\textit{average value}} is close to the true value. \\
			\midrule
			Vector Quantity & A quantity that has a \textbf{\textit{magnitude and direction}}. \\
			Scalar Quantity & A quantity that has a magnitude only. \\
			\bottomrule
		\end{tabular}
	\end{center}
	
	\section{Kinematics}
	We define a coordinate system with defined reference positive directions and we assume constant acceleration.
	\begin{center}
		\renewcommand{\arraystretch}{1.2}
		\begin{tabular}{@{} l l p{8.5cm} @{}}
			\toprule
			Displacement & $\textbf{s}$ & The distance travelled in a stated direction from a reference point. \\
			Velocity & $\displaystyle \textbf{v} = \frac{d\textbf{s}}{dt}$ & The rate of change of displacement with respect to time.\\
			\rule{0pt}{20pt}Speed & $\displaystyle v=\left|\textbf{v}\right| = \left|\frac{d\textbf{s}}{dt}\right|$ & The rate of change of distance travelled with respect to time. \\
			\rule{0pt}{20pt}Acceleration &  $\displaystyle \textbf{a} = \frac{d\textbf{v}}{dt} = \frac{d^2\textbf{s}}{dt^2}$ & The rate of change of velocity with respect to time. \\
			\bottomrule
		\end{tabular}
	\end{center}
	
	\section{Dynamics}
		\subsection{Newton's Laws of Motion}
			\oldparskip=\parskip
			\begin{center}
				\renewcommand{\arraystretch}{1.2}
				\begin{tabular}{@{} l >{\parskip=\oldparskip}p{10.5cm} @{}}
					\toprule
					1\textsuperscript{st} Law & A body will continue in its \textit{\textbf{state of rest}}, or \textbf{\textit{move}} at \textbf{\textit{constant speed in a stright line}} unless an \textbf{\textit{external resultant force}} acts on it.\\
					$\rightarrow$ Inertia & The resistance to change in the state of motion of an object \\
					$\rightarrow$ Mass & A property of that determines the objects inertia. \\
					\midrule
					2\textsuperscript{nd} Law & The \textbf{\textit{rate of change of linear momentum}} of a body is \textbf{\textit{directly proportional}} to the resultant force acting on it, and its direction is in the \textbf{\textit{same direction}} as this resultant force. \par The \textbf{\textit{force acting on an object}} is defined as the \textbf{\textit{rate of change of linear momentum}} of an object. $$F \propto \frac{dp}{dt}, ~F=ma \textrm{ (if constant mass)} \vspace*{-\baselineskip}$$ \\
					\midrule
					3\textsuperscript{rd} Law & If body A exerts a force on body B, then body B will exert an \textbf{\textit{equal and opposite}} force on body A. \par \textit{Note:} Action-Reaction Pairs act on different bodies and are of the same nature.\\
					\midrule
					Weight & The gravitational force acting on the object. \\
					Weightlessness & There is no contact force acting on the object. \textit{A body experiences apparent weightlessness when the resultant force acting on it is its weight, or it is undergoing freefall.} \\
					\bottomrule
				\end{tabular}
			\end{center}
		\subsection{Momentum}
			\begin{center}
				\renewcommand{\arraystretch}{1.2}
				\begin{tabular}{@{} p{2.7cm} l p{6.8cm} @{}}
					\toprule
					Linear Momentum & $\textbf{p}=m\textbf{v}$ & The product of an object's mass and its velocity. \\
					\rule{0pt}{20pt}Impulse & $\displaystyle \textbf{J}=\int_{t1}^{t2}\textbf{F}~dt=P_{f}-P_{i}$	& The product of the average force acting on an object and the time interval that the force is being applied.\\	
					\midrule
					Principle of Conservation of Linear Momentum & \multicolumn{2}{p{10.7cm}}{The total momentum of the system is a constant when no external resultant force acts on it.}\\
					\bottomrule
				\end{tabular}
			\end{center}
	\newpage
	\section{Forces}
		\begin{center}
			\renewcommand{\arraystretch}{1.2}
			\begin{tabular}{@{} l l p{7cm} @{}}
				\toprule
				Pressure due to Fluid & $\Delta P = h\rho g$ & The force acting per unit area by the fluid on a body submerged at a depth in the fluid. \\
				Upthrust & $U=m_{f}g=\rho V_{dis}g$ & The \textbf{\textit{net force exerted by a fluid}} on a body submerged in the fluid. \\
				Principle of Floatation & $mg=U=\rho V_{dis} g$ & This holds true for an object floating in equillibrium in a fluid.  \\
				Drag & \multicolumn{1}{p{2.5cm}}{$\textbf{F}_{\textbf{D}}=k\textbf{v}$ \par (Laminar Flow)} & It is the force resisting an object \textbf{\textit{moving relative to a fluid}}. It always \textit{\textbf{opposes}} motion, and its magnitude is \textbf{\textit{dependent on the velocity}} of the object.\\
				\midrule
				Moment of a force (Torque) & $\boldsymbol{\tau}=\textbf{r} \times \textbf{F}$ & Moment of a force about a point (the pivot) is the \textit{\textbf{product}} of the magnitude of the force and the \textbf{\textit{perpendicular distance}} of the \textbf{\textit{line of action}} of the force to the point. \\
				Couple & \multicolumn{2}{p{10.3cm}}{A couple always consists of 2 parallel forces which are equal in magnitude and opposite in direction (their lines of action fo not coincide)} \\
				Torque of a couple & \multicolumn{2}{p{10.3cm}}{The \textbf{\textit{product}} of the \textbf{\textit{magnitude of one of the forces}} of the couple and the \textbf{\textit{perpendicular distance between the forces.}}} \\
				Center of gravity of a body & \multicolumn{2}{l}{It is the point at which the weight of the body appears to act.} \\
				\bottomrule
			\end{tabular}
		\end{center}
		\subsection{Equilibrium of Forces}
			For a rigid body to be in static equilibrium, 2 conditions must be satisfied:
			
			\begin{tabular}{@{} C{0.5\textwidth} C{0.5\textwidth}  @{}}
				1. \textbf{Translational equilibrium} \par The \textbf{net external} force acting on the body is zero. $$\sum F = 0$$ & 2. \textbf{Rotational Equilibrium} \par The \textbf{net torque} on the body about \textbf{\underline{ANY} point} is zero. $$\sum \tau = 0$$ \vspace*{-\baselineskip} \\
			\end{tabular}
			
			For a 3-forces system in static equilibrium, the 3 forces for \textit{a closed vector triangle}. For 3 forces acting on an \textit{extended body} in static equilibrium, the lines of action of the 3 forces \textit{must intersect at a common point} unless the 3 forces are parallel.
	\section{Work, Energy, and Power}
	\section{Circular Motion}
	\section{Gravitation}
	\section{Oscillations}
	\section{Waves}
	\section{Superposition}
	\section{Thermal Physics}
		\begin{center}
			\renewcommand{\arraystretch}{1.2}
			\begin{tabular}{@{} l l p{8.5cm} @{}}
				\toprule
				Temperature & $T$ & The average kinetic energy the molecules in a system possess. \\
				Heat & $Q$ & Transfer of thermal energy from regions of higher to lower temperature. \\
				Thermal Equilibrium & & 2 objects in thermal contact with no net exchange of heat. \\
				Kelvin Scale & & Absolute temperature scale independent of thermometric properties. \\
				Absolute Zero & 0K & All molecules possess minimal internal energy. \\
				Specific Heat Capacity & C & Amount of thermal energy per unit substance to increase the temperature of the unit substance by one unit of temperature. \\
				Specific latent heat of fusion & $L_f$ & Amount of \textbf{thermal energy per unit mass} to convert the substance from solid to liquid without any change in temperature. \\
				Specific latent heat of vaporisation & $L_v$ & Amount of \textbf{thermal energy per unit mass} to convert the substance from liquid to gas without any change in temperature. \\
				Internal Energy & $U$ & Sum of microscopic random kinetic energy and microscopic potential energy of molecules in system. For ideal gases: $$U=\frac{3}{2}~nRT$$ \vspace*{-\baselineskip} \\ 
				\bottomrule
			\end{tabular}
		\end{center}
		\subsection{Laws of Thermodynamics}
			\begin{center}
				\renewcommand{\arraystretch}{1.2}
				\begin{tabular}{@{} l p{11cm} @{}}
					\toprule
					0\textsuperscript{th} & If two systems are in thermal equilibrium with a third system, they are in thermal equilibrium with each other. \\
					1\textsuperscript{st} & Increase in internal energy of system is sum of heat absorbed by system and work done on system. $$\Delta U = Q + W_{on}$$ \vspace*{-\baselineskip} \\
					\bottomrule
				\end{tabular}
			\end{center}
			We need not know the 3\textsuperscript{rd} and 4\textsuperscript{th} laws.
		\subsection{PV Graphs}
			We assume, for the following, that the arrow points towards the positive-$V$ direction. 
			\begin{center}
				\renewcommand{\arraystretch}{1.2}
				\begin{tabular}{@{} l l l @{}}
					\toprule
					Isobaric & Constant Pressure  & $W_{on} < 0 ~,~~ \Delta U > 0$\\
					Isochoric & Constant Volume & $W_{on} = 0 ~,~~ \Delta U > 0$\\
					Isothermal & Constant Temperature & $W_{on} < 0 ~,~~ \Delta U = 0$\\
					Adiabatic & Thermally Insulated & $W_{on} < 0 ~,~~ Q = 0$\\
					Cyclic & Start and end at the same state & $\Delta U = 0$\\
					\bottomrule
				\end{tabular}
			\end{center}
	\section{Electric Fields}
	\begin{center}
		\renewcommand{\arraystretch}{1.2}
		\begin{tabular}{@{} l l p{8.5cm} @{}}
			\toprule
			Electric Field & $\displaystyle E=\frac{Q}{4\pi \varepsilon_0 r^2}$ & Electric force per unit charge acting on small positive test charge at that point \\
			Coulomb's Law & $\displaystyle F_E=\frac{|Q_1||Q_2|}{4\pi \varepsilon_0 r^2}$ & Magnitude of the electric force between 2 point charges is directly proportional to the product of the magnitude of their charges and inversely proportional to square of their distance\\
			\multicolumn{3}{c}{$F_E=qE$} \\
			\midrule
			Electric Potential & $\displaystyle V = \frac{Q}{4\pi \varepsilon_0 r}$ & \textbf{Work done per unit charge} by external force to bring small positive test charge from infinity to that point in an electric field without change in kinetic energy \\
			Electric Potential Energy & $\displaystyle U = \frac{Q_1 Q_2}{4\pi \varepsilon_0 r}$ & \textbf{Work done } by external force to bring small positive test charge from infinity to that point in an electric field without change in kinetic energy\\
			\multicolumn{3}{c}{$U=qV$} \\
			\bottomrule
		\end{tabular}
	\end{center}
	
	\section{Current of Electricity}
	\section{DC Circuits}
	\section{EM and EMI}
	\section{Alternating Current}
	\section{Quantum Physics}
	\section{Lasers and Semiconductors}
	\section{Nuclear Physics}
	
\end{document}
